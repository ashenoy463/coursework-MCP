\begin{Large}
    \textsf{\textbf{Program 4 - Gaussian Integral by Monte Carlo Integration}}
\end{Large}

\vspace{2em}

\header{Aim:} To estimate the value of a gaussian integral to within a given precision using uniform-sample Monte Carlo integration along with ensemble-averaging\\

Consider an integral that is perhaps analytically intractable:

$$
    I = \int_a^b dx\ f(x)
$$

The general Monte-Carlo integration problem is to write $I$ in the form

$$
    I = \int_a^b dx\ g(x)h(x)dx
$$

where is a probablity distribution $g(x)$ we have an efficient method to sample from. We can then note that the integral we wish to solve is simply the expectation value of the function $h(X)$ where $X$ is distributed as $g(x)$. This expectation value can now be estimated by drawing say $N$ samples $\xi_i$ from $g(x)$, thus yielding

$$ 
    I = \langle h \rangle \approx \frac{1}{N} \sum_{i=1}^N h(\xi_i)
$$

In our case we wish to solve the integral

$$
    I_\alpha = \int_0^t dx\ e^{-\alpha x^2}
$$

Applying the above procedure, we can insert a uniform distribution (scaled from $0$ to $t$), $U(x)= 1/t$ and obtain

$$
    I = \frac{t}{N}\sum_{i=1}^N e^{-\alpha \xi_i ^2}
$$

To compute the error in estimation, we note that the well-tabulated error-function is given by

\begin{align*}
    \text{erf}(t) &= \frac{2}{\sqrt{\pi}} \int_0^t dx\ e^{-\alpha x^2} \\
    \implies I_{\alpha} &= \sqrt{\frac{\pi}{4\alpha}} \text{erf}(t) 
\end{align*}

\newpage

\header{Flowchart:}


% \begin{figure}[!htb]
%     \centering
%     \begin{tikzpicture}[node distance=2cm]
%         \node (start) [startstop] {Start};
%     \end{tikzpicture}
% \end{figure}

\newpage

\small\texttt{\verbatiminput{code/4_mcinteg.f90}}

\newpage

\header{Input:}

\small\texttt{\verbatiminput{io/mcinteg.in}}

\header{Output:}

\small\texttt{\verbatiminput{io/mcinteg.out}}
