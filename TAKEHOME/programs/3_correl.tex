\begin{Large}
    \textsf{\textbf{Question 3}}
\end{Large}

\vspace{2em}

We are tasked with evaluating the expectation value of the Coulomb interaction in the ground state of the Helium atom, expressed in spherical polar coordinates: 
% \huge\langle \frac{1}{\abs{\Vec{r}_1 - \Vec{r}_2}} \LARGE\rangle

$$
    I = \int_{r_1=0}^{\infty} \int_{r_2=0}^{\infty} \int_{\theta_1=0}^{\pi} \int_{\theta_2=0}^{\pi} \int_{\phi_1=0}^{2\pi} \int_{\phi_2=0}^{2\pi} r_1^2 e^{-4r_1} r_2^2 e^{-4r_2} \sin{\theta_1}\sin{\theta_2}\frac{1}{\sqrt{r_1^2 + r_2^2 - 2 r_1 r_2 \cos\beta}} dr_1 dr_2d\theta_1 d\theta_2 d\phi_1 d\phi_2
$$

where 

$$
    \cos{\beta} = \cos{\theta_1} + \cos{\theta_2} + \sin{\theta_1}\sin{\theta_2}\cos{(\phi_1 - \phi_2)}
$$

We note that the integral has a sharp inverse exponential dependence on $r_1$ and $r_2$, making exponential sampling desirable for the radial co-ordinates. For the angular co-ordinates, we use uniform sampling over their domain.

To sample from the exponential distribution, we use the Inverse-CDF method. Consider the PDF

$$ 
f_X (x) =
\begin{dcases} 
    4e^{-4x}     & x \ge 0  \\
    0           & \text{otherwise} 
\end{dcases}
$$

The corresponding CDF is

$$ 
F_X (x) =
\begin{dcases} 
    1 - e^{-4x}     & x \ge 0  \\
    0           & \text{otherwise} 
\end{dcases}
$$

The inverse CDF in terms of a uniformly distributed random variable $Y$ is then


$$ 
F^{-1}_Y (y) =
\begin{dcases} 
    -\frac{1}{4}\ln{(1-y)}     & 0 \le y \le 1   \\
    0           & \text{otherwise} 
\end{dcases}
$$

Since all the random variables are independent of each other, the joint PDF is simply the product of the individual PDFs:

$$
f(r_1,r_2,\theta_1,\theta_2,\phi_1,\phi_2) =  \frac{4}{\pi^4} e^{-4(r_1 + r_2)}
$$

Define 

$$
g(r_1,r_2,\theta_1,\theta_2,\phi_1,\phi_2) =  \frac{r_1^2 r_2^2 \sin{\theta_1}\sin{\theta_2}}{\sqrt{r_1^2 + r_2^2 - 2 r_1 r_2 \cos\beta}}
$$

The integral is then (with aforementioned limits implied)

\begin{align*}
    I &= \frac{\pi^4}{4}\int f(r_1,r_2,\theta_1,\theta_2, \phi_1,\phi_2) g(r_1,r_2,\theta_1,\theta_2,\phi_1,\phi_2) dr_1 dr_2d\theta_1 d\theta_2 d\phi_1 d\phi_2  \\
      &= \frac{\pi^4}{4} \langle g\rangle
\end{align*}




\newpage

\header{Code:}

\small\texttt{\verbatiminput{code/correl.f90}}

\newpage

\header{Output:}

\small\texttt{\verbatiminput{io/correl.out}}

\begin{figure}[!htb]
    \centering
    \includegraphics[width=1.0\linewidth]{diag/error.png}
    \caption{Error Analysis}
\end{figure}